%%%%%%%%%%%%%%%%%%%%%%%%%%%%%%%%%%%%%%%%%
% Beamer Presentation
% LaTeX Template
% Version 1.0 (10/11/12)
%
% This template has been downloaded from:
% http://www.LaTeXTemplates.com
%
% License:
% CC BY-NC-SA 3.0 (http://creativecommons.org/licenses/by-nc-sa/3.0/)
%
% Modified by Jeremie Gillet in November 2015 to make an OIST Skill Pill template
%
%%%%%%%%%%%%%%%%%%%%%%%%%%%%%%%%%%%%%%%%%

%----------------------------------------------------------------------------------------
%	PACKAGES AND THEMES
%----------------------------------------------------------------------------------------

\documentclass{beamer}

\mode<presentation> {

\usetheme{Madrid}

\definecolor{OISTcolor}{rgb}{0.65,0.16,0.16}
\usecolortheme[named=OISTcolor]{structure}

%\setbeamertemplate{footline} % To remove the footer line in all slides uncomment this line
%\setbeamertemplate{footline}[page number] % To replace the footer line in all slides with a simple slide count uncomment this line

\setbeamertemplate{navigation symbols}{} % To remove the navigation symbols from the bottom of all slides uncomment this line
}

\usepackage{graphicx} % Allows including images
\usepackage{booktabs} % Allows the use of \toprule, \midrule and \bottomrule in tables
\usepackage{textpos} % Use for positioning the Skill Pill logo
\usepackage{tikz} % Allow modification of transparency for title slide image

% For code displays
\usepackage{listings}
\usepackage{color}
\usepackage{amsmath}
\usepackage{listing}

\usepackage{array}

% Allows disabling 'ligatures' (--option with two dashes instead of becoming em-dash/en-dash)
\usepackage{microtype} 
\DisableLigatures[-]{}


\definecolor{dkgreen}{rgb}{0,0.6,0}
\definecolor{gray}{rgb}{0.5,0.5,0.5}
\definecolor{mauve}{rgb}{0.58,0,0.82}

\lstset{frame=tb,
  language=python,
  aboveskip=3mm,
  belowskip=3mm,
  showstringspaces=false,
  columns=flexible,
  basicstyle={\small\ttfamily},
  numbers=none,
  numberstyle=\tiny\color{gray},
  keywordstyle=\color{blue},
  commentstyle=\color{dkgreen},
  stringstyle=\color{mauve},
  breaklines=true,
  breakatwhitespace=true,
  tabsize=3
}



%----------------------------------------------------------------------------------------
%	TITLE PAGE
%----------------------------------------------------------------------------------------

\title[Skill Pill]{Skill Pill: Introduction to Git and Version Control} % The short title appears at the bottom of every slide, the full title is only on the title page
\subtitle{Lecture 2: Git it on!}

\author[Christian Butcher] % Shows on every footer
	{\textbf{Valentin Churavy - 2016} \\ James Schloss - 2018 \\ Christian Butcher - 2019, 2020} % Title slide names
\institute[OIST] % Your institution as it will appear on the bottom of every slide, may be shorthand to save space
{
Okinawa Institute of Science and Technology \\ % Your institution for the title page
\textit{christian.butcher@oist.jp} % Your email address
}
\date{November 26, 2020} % Date, can be changed to a custom date

\begin{document}


\setbeamertemplate{background}{
\begin{tikzpicture}\node[opacity=0.6]{\includegraphics[width=\paperwidth]{SPbackground.png}}; % Adding the background logo
\end{tikzpicture}}

\begin{frame}
\vspace*{1.4cm}
\titlepage % Print the title page as the first slide
\end{frame}

\setbeamertemplate{background}{} % No background logo after title frame

\addtobeamertemplate{frametitle}{}{% Adding the Skill Pill logo on the title screen after title frame
\begin{textblock*}{100mm}(.8\textwidth,-1.25cm)
\includegraphics[height=2cm]{SPwhite.png}
\end{textblock*}}


\begin{frame}
\frametitle{Overview} % Table of contents slide, comment this block out to remove it
\tableofcontents % Throughout your presentation, if you choose to use \section{} and \subsection{} commands, these will automatically be printed on this slide as an overview of your presentation
\end{frame}

%----------------------------------------------------------------------------------------
%	PRESENTATION SLIDES
%----------------------------------------------------------------------------------------

%------------------------------------------------

\section{Recap}
\begin{frame}{Recap}
  Last week we covered (don't forget to prefix with \textbf{git}):
  \begin{itemize}
    \item \textbf{clone} : Cloning a repository into a new directory
    \item \textbf{add} : Add file contents to the \textit{index}. This makes git track the file.
    \item \textbf{commit} : Record changes. Store the staged files as a new part of the history!
    \item \textbf{push} : Update remote \textit{refs} and objects.
	\end{itemize}
	There\rq{}s also a \textbf{pull} command.
	\begin{itemize}
		\item \textbf{pull} : Update from a \textit{remote}. Technicially a combination of \textbf{fetch} and \textbf{merge} by default.
	\end{itemize}
\end{frame}

\begin{frame}{Some definitions/descriptions}
	\renewcommand{\arraystretch}{2.5}
	\begin{tabular}{m{0.2\textwidth}m{0.6\textwidth}}
	  \textit{remote} & Another git repository. \newline We used GitHub to provide this. \\
	  \hline
  		\textit{index} & A single, large, binary file listing all files in the current branch with some extra information. \newline Reflects the \lq\lq{}proposed next commit\rq\rq{} \\
  		\hline
	  \textit{refs} & Short for references. \newline Can point to almost anything in git.
  \end{tabular}
  	\renewcommand{\arraystretch}{1}
\end{frame}

\begin{frame}{Tips so far...}
  \begin{itemize}
    \item You can use \texttt{git help <command>} or \texttt{git <command> --help} to get information about a command, like \texttt{clone}.
    \item \texttt{git add \textbf{-p}} uses \textit{patch mode} to interactively add parts of a file. \textbf{-i} is interactive without patch mode.
    \item \texttt{git rm} can be used to remove files from the index (and optionally working directory), whilst \texttt{git mv} can help you move files within the repository.
    \item \texttt{git commit --amend} opens an editor to alter the previous commit's message. Don't do this if you already \textbf{push}ed the commit!
  \end{itemize}
\end{frame}

\begin{frame}{Reset and Checkout}
  We also considered \textbf{reset} and \textbf{revert}.
  \begin{itemize}
    \item Reset is a fairly complicated tool, which modifies the three \lq{}trees\rq{} we have briefly mentioned/considered - \texttt{HEAD} (your last commit), \texttt{index} (the staging area) and the \texttt{working directory}.
    \item If you\rq{}re interested to know more about this tool, there is a long and informative guide at \url{https://git-scm.com/book/en/v2/Git-Tools-Reset-Demystified}.
    \item This content is really beyond the scope of our Skill Pill. :(
  \end{itemize}
\end{frame}


\section{Working with Remotes}
\subsection{Remotes}
\begin{frame}{Remotes}
Last week we introduced \textbf{GitHub}. GitHub is a service that offers you a solution to remotely store your repositories.
\begin{itemize}
    \item Git is \emph{Distributed} Version Control System (DVCS). Every copy of your repository, may it be remote or local, is independent of each other. There is no central master repository. 
    \item In order to synchronize these distributed copies we introduce the concept of a remote.
  \begin{block}{}
    git \textbf{remote}
  \end{block}
  \item There can be as many remotes as you want each with different names. When you clone a repository there will be one default remote called \textbf{origin}.
\end{itemize}
\end{frame}

%\begin{frame}
%  \begin{block}{Exercise 1a}
%    \begin{enumerate}
%      \item Tell Jeremie your GitHub user name so you can be added to the allowed committers for the repository...
%      \item Clone this repository from GitHub: \url{https://github.com/oist/skillpill-git-group1}
%      \item Add your favourite color to the colors file.
%      \item Commit your change with an appropriate message, but don\rq{}t \textbf{push}!
%    \end{enumerate}
%  \end{block}
%\end{frame}
%
%\begin{frame}
%  \begin{block}{1b: Alternative Exercise if I can\rq{}t get access to the repository...}
%    \begin{enumerate}
%      \item Create an empty directory for your repositories (note plural)
%      \item In this directory, \texttt{A}, create a new directory, called \lq{}myremote\rq{} (or change as needed below).
%      \item Change into this new directory, and run \texttt{git init --bare}. This creates a \lq{}bare\rq{} repository, in which nothing is checked out.
%      \item Move back into directory A. Create a new directory, and then run \texttt{git init} (not bare), add a file, and commit it.
%      \item Run the command \texttt{git remote add origin ../myremote/}, which adds a remote repository called \lq{}origin\rq{}, pointing to the directory \lq{}myremote\rq{}.
%    \end{enumerate}
%  \end{block}
%\end{frame}
%\begin{frame}
%  \begin{block}{1b: Alternative Exercise Continued...}
%    \begin{enumerate}
%      \setcounter{enumi}{5} % Continue counting from 6 (5 points in previous slide)
%      \item Push your commit with the command \texttt{git push -u origin master}, which sets your branch to track the new branch \lq{}master\rq{} on \lq{}origin\rq{}.
%      \item Go back to A. Clone your \lq{}myremote\rq{} repository with the command \texttt{git clone myremote secondcopy} (a new directory named \lq{}secondcopy\rq{} will be created).
%      \item Make changes to your file in both your first repository, and \lq{}secondcopy\rq{} (not \lq{}myremote\rq{}). Commit both. Push from only the first repository (not \lq{}secondcopy\rq{}).
%    \end{enumerate}
%  \end{block}
%\end{frame}

\subsection{Branches}
\begin{frame}{Branches}
  Since there git is decentralized there is no one state of the repository that is correct. To manage this complexity git has the notion of a branch. 
  \begin{block}{}
    \begin{itemize}
      \item git \textbf{branch}  Manages branches. 
      \item git \textbf{checkout} Switch between branches.
    \end{itemize}
  \end{block}

  \begin{itemize}
    \item Most repositories have a default branch called \textbf{master} (this is changing to "main" in some cases). Branches are just names for points in the history.
    \item Once we start working with branches we have to ask ourselves how are we going to join them back up? We can do this by performing a merge.
    \item You can also associate a local branch with a remote branch by setting it as upstream. git push \textbf{-u}.
  \end{itemize}
\end{frame}

\begin{frame}
  \begin{block}{Exercise}
    \begin{enumerate}
      \item Create a new branch, based of master
      \item Add a few commits to your branch
      \item Change back onto master
      \item Check the contents of the file(s) you changed on your other branch whilst you\rq{}re on the master branch
    \end{enumerate}
  \end{block}
\end{frame}

\subsection{Merging}
\begin{frame}{Merging - An Introduction}
  \begin{itemize}
    \item We perform \textit{merges} to \lq\lq{}join two or more development histories together\rq\rq{}.
    \item It is most commonly performed invisibly by \texttt{git pull} and performs by default a \lq\lq{}fast-forward\rq\rq{} merge.
    \item We usually see this first when we try to pull some changes and we cannot perform a fast-forward merge.
    \item In that case, we have to resolve the \textit{merge conflict}.
  \end{itemize}
\end{frame}

\begin{frame}[fragile]{Merging}
Merging is the act of joining two branches together or to join two different branches. You will always merge \emph{from} a branch/remote into a branch.
  \begin{block}{}
    \begin{itemize}
      \item git \textbf{fetch}  Gets remote changes
      \item git \textbf{merge}  Merge changes (ff by default)
      \item git \textbf{add}    Resolve merge-conflict
    \end{itemize}

    Options for merge:
    \begin{description}
      \item[--no-commit] Performs the merge, but doesn't commit yet. Gives you a chance to edit the merge commit.
      \item[--ff-only]   Aborts when we can't perform a fast-forward merge.
      \item[--abort]     Aborts current conflict-resolution and reset to previous state.
    \end{description}
  \end{block}

  You can visualize your history in many different ways, but a nice way from the command line is.\\
  \begin{block}
  git log \textbf{--graph --decorate --oneline --all}
  \end{block}
\end{frame}

\begin{frame}
  \begin{block}{Exercise}
    \begin{enumerate}
    	\item \textbf{Clone} a fork of the repository at \url{https://github.com/oist/skillpill-git} (you may have this available from last week)
		\item \textbf{Checkout} the `merge-main' branch
		\item \textbf{Merge} the `merge-AddNameToGreeting' branch. Optionally use ``--no-ff'' to force a merge commit. This will succeed without conflict.
		\item Attempt to merge the `merge-TimeOfDayGreeting' branch. This will cause a merge conflict!
	\end{enumerate}
  \end{block}
\end{frame}

\begin{frame}[fragile]
	\begin{lstlisting}[language=Python, title={Merge conflict contents}, basicstyle=\tiny]
def main(username, timeValue):
<<<<<<< HEAD
    print("Hello " + username)

def callFctn(args):
    if len(args) > 1:
        username = args[1]
    else:
        username = "World"

    timeValue = ""
=======
    greeting = getGreeting(timeValue)
    print(greeting + " " + username)

def callFctn(args):
    username = "World"
    
    if len(args) > 1:
        try:
            timeValue = time.strptime(args[1], "%H:%M")
        except:
            timeValue = ""
    else:
        timeValue = ""

>>>>>>> merge-TimeOfDayGreeting
	\end{lstlisting}
\end{frame}

\begin{frame}
  \begin{block}{Exercise continued}
    \begin{enumerate}
    	\setcounter{enumi}{4}
	    \item Use a text editor to resolve the conflict
		\item Commit the resolved file (don't forget to \textbf{add})
		\item Push your branch to your forked repository
	\end{enumerate}
  \end{block}
  This brings us on to ``Pull Requests''...
\end{frame}


%\begin{frame}[fragile]{Interlude: .gitconfig}
%  The file \emph{.gitconfig} can be used to set default options per user or per project. The user files is in \emph{\textasciitilde/.gitconfig}. Each option can also be set with git \textbf{config}.
%  \begin{lstlisting}
%  [user]
%    email = v.churavy@gmail.com
%    name  = Valentin Churavy
%  [github]
%    user = vchuravy
%  [push]
%    default = simple
%  [rerere]
%    enabled = true
%  \end{lstlisting}
%\end{frame}


\section{Pull Requests on GitHub}
\begin{frame}{Pull Requests}
  \begin{itemize}
    \item Pull Requests are a GitHub-specific feature (also implemented on other platforms, but not a git feature) used to allow contributing code to a repository.
    \item They are typically used when you don\rq{}t have write access to a repository
    \item They can also be used to allow review of your code, perhaps by a coworker, even if you could directly push your changes
    \item Without using extensions, you must use the website to use them
  \end{itemize}
\end{frame}

\begin{frame}
  \begin{block}{Demo + Exercise}
    \begin{itemize}
      \item Demonstration...
      \item Practice:
      \begin{enumerate}
        \item In the last exercise we pushed commits to forks of the OIST repository
        \item Open a pull request on GitHub against the original repository
      \end{enumerate}
    \end{itemize}
  \end{block}
\end{frame}

\section{More Advanced Topics}
\subsection{Rebasing and Rewriting history}
\begin{frame}[fragile]{Rewriting History}
  Rebases are a way to create fast-forward merges, by altering \emph{history}. Each branch has a root commit from which it diverged from the original commit. By rebasing we change this root. This has a couple of side effects. 
  \begin{itemize}
    \item Linear commit history.
    \item No merge commits within a branch.
    \item commit-ids change.
  \end{itemize}

  \begin{block}{}
    \begin{itemize}
      \item git \textbf{pull --ff-only} Don't merge if there are conflict with the remote
      \item git \textbf{rebase} Perform a rebase
      \item git \textbf{rebase -i} Perform a interactive rebase
      \item git \textbf{push -f} Force push your changes
      \item git \textbf{pull --rebase} Perform a pull with a rebase
    \end{itemize}
  \end{block}
\end{frame}

\begin{frame}
  \begin{block}{Exercise}
    \begin{enumerate}
      \item create a branch, with some commits
      \item go back to master and do some additional work
      \item rebase your branch onto master
      \item merge your branch onto master
    \end{enumerate}
  \end{block}
\end{frame}

\begin{frame}[allowframebreaks,fragile]{Secrets!}
 \begin{block}{Stash}
    When you are moving between branches you sometines want to keep your non-commited changes associated with the branch you where doing them one.
    \begin{itemize}
      \item git \textbf{stash}
      \item git \textbf{stash pop}
    \end{itemize}
  \end{block}

  \begin{itemize}
    \item git commit \textbf{--amend} Amend the last commit.
    \item git add \textbf{-i} Interactive add
    \item git add \textbf{-p} Interactive add in patch mode.
    \item git \textbf{rm} Removes file.
    \item git \textbf{mv} Move file within repository
  \end{itemize}
  
  \framebreak
    
  \begin{block}{Autosquash}
    \begin{itemize}
      \item git config rebase.autosquash true
      \item git commit \textbf{--squash=}\emph{some-hash}
      \item git commit \textbf{--fixup=}\emph{some-hash}
     \end{itemize}
     Autosquash will reorder the commits appropriatly before you perform a git \textbf{rebase -i}.
  \end{block}
  \begin{block}{Blame}
    There is no such thing as \textit{good} code. If you are using git with people, chances are that something will break at some time and you need someone to blame. That's what git blame is for:
    \begin{lstlisting}
        git blame -L 1,3 file
    \end{lstlisting}
  \end{block}
\end{frame}

\end{document}
